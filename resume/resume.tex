\documentclass[
	11pt
]{resume}

\usepackage{fontspec}
\usepackage{hyperref}
\setmainfont{Times New Roman}

\name{Ankur Desai}
\address{(248) $\cdot$ 657 $\cdot$ 3805 \\ ardusa05@gmail.com \\ \href{https://ardusa.github.io/}{https://ardusa.github.io}}

\begin{document}
	\begin{rSection}{E}{ducation}
		\begin{rSectionEntry}{Michigan State University}{Expected, May 2028}{B.S. in Computer Science \& Engineering}{East Lansing, MI}
			\item GPA: 3.74
			\item Relevant Coursework: Linear Algebra, Discrete Mathematics, Computer Organization \& Architecture (Spring 2026), Object-Oriented Software Design (Spring 2026)
		\end{rSectionEntry}
	\end{rSection}

	\begin{rSection}{E}{xperience}
		\begin{rSectionEntry}{Drone Systems Research}{September 2025 - Present}{Exedy Drones}{East Lansing, MI}
			\item Researching flight path simulations and path making algorithms for autonomous agricultural drones using Python
			\item Building object detection models using TensorFlow and OpenCV for drone navigation and obstacle avoidance
		\end{rSectionEntry}

		\begin{rSectionEntry}{Web Development Intern}{May 2025 - August 2025}{WFS Consulting Group}{Plymouth, MI}
			\item Crafted a Figma design and implemented responsive and dynamic elements using CSS media queries and JavaScript
			\item Deployed the website via GoDaddy Linux hosting servers, ensuring live client access and functionality worldwide
			\item Drove customer engagement by 200\% through integrating contact forms, social media links, and interactive elements
		\end{rSectionEntry}
	\end{rSection}

	\begin{rSection}{P}{rojects}
		\begin{rSectionEntry}{WizViz (SpartaHack X)}{February 2025}{Hackathon Project | Python, OpenCV, MediaPipe, PyGame}{\href{https://devpost.com/software/wizviz}{Devpost Link}}
			\item Developed an augmented reality game using Computer Vision and PyGame where two players can duel as wizards, casting spells, dodging attacks, and using healing potions, all in real-time via hand gestures and body movements
			\item Awarded Best Game in Interactive Media track, among 100+ projects and 350+ participants at SpartaHack X
			\item Applied OpenCV and MediaPipe Pose model to track multiple player movements, enabling gesture recognition
		\end{rSectionEntry}

		\begin{rSectionEntry}{Project Team Lead}{September 2025 - Present}{ThinkStack}{East Lansing, MI}
			\item Orchestrated a cross-functional team of 4 software engineers and 2 designers to architect, launch, and iterate upon a Next.js webapp and an Electron desktop application for note-taking, concept visualization, and creative exploration
			\item Led SCRUM meetings, built CI/CD pipelines, and drove collaboration between frontend, backend, and UI/UX teams
			\item Developed a REST API using FastAPI, SQLAlchemy, MongoDB, and Neo4j to handle authentication, manage user data, efficiently store and query notes, and visualize relationships between concepts and notes
			\item Used requirements engineering to gather client feedback, define project scope, and plan features for development
		\end{rSectionEntry}

		\begin{rSectionEntry}{Mira - AI Personal Assistant}{May 2025 - September 2025}{Full-Stack Development | Python, FastAPI, SQL, Electron, Node.js}{\href{https://github.com/mira-assistant}{GitHub Link}}
			\item Developed a REST API using FastAPI, Whisper, noisereduce, DBScan, and Gemini to transcribe audio, assign speaker IDs, extract actionable tasks, and enable conversational interactions and voice commands with Mira
			\item Designed an Electron desktop client that listens passively for voice commands and maintains interaction history
			\item Implemented a client-selection algorithm with Voice Activity Detection (VAD) to support multiple devices on each Mira Network, and intelligently switching between connected clients based on proximity, uptime, and noise level
			\item Deployed the API on AWS Lambda, and utilized Amazon RDS and SQLAlchemy to store user data and interactions
		\end{rSectionEntry}
	\end{rSection}

	\begin{rSection}{T}{echnical Skills}
		\begin{rSet}{Languages}{Java, Python, JavaScript, SQL, Bash}
		\end{rSet}
		\begin{rSet}{Frameworks \& Libraries}{React, Node.js, Electron, FastAPI, SQLAlchemy, MediaPipe, OpenCV, Gemini API}
		\end{rSet}
		\begin{rSet}{Databases}{MongoDB, Neo4j, PostgreSQL, Amazon RDS}
		\end{rSet}
		\begin{rSet}{Tools \& Platforms}{Docker, Git, GitHub Actions, Cursor IDE, Figma, AWS Lambda, GoDaddy Hosting}
		\end{rSet}
	\end{rSection}

\end{document}
